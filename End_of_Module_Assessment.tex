\documentclass[12pt]{article}
\usepackage{fancyhdr,a4wide}
\usepackage[mathscr]{eucal}
\usepackage[scaled]{helvet}
\renewcommand\familydefault{\sfdefault} 
\usepackage[T1]{fontenc}
\usepackage{blindtext}
\usepackage{graphicx}
\usepackage{amscd}
%\usepackage{xypic}
\usepackage{amsfonts}
\usepackage{amsmath}
\usepackage{amssymb}
%\usepackage{latexsymb}
\usepackage{lastpage}
\setlength{\headheight}{15.2pt}
\pagestyle{fancy}
\setlength{\textwidth}{16truecm}
\setlength{\textheight}{24truecm}
\setlength{\oddsidemargin}{3mm}
\setlength{\evensidemargin}{3mm}
\setlength{\topmargin}{1mm}
\usepackage{setspace}
%\usepackage{cmbright} %PUTS TEXT AND EQUATIONS INTO SANS SERIF FONT

\fancyhf{}
\hoffset=-4mm
\voffset=-20mm
\parindent=0mm
\parskip=2mm

\def\semester{Semester Two}
\def\session{2020/2021}
\def\code{PHYSXXXX} %PUT CORRECT MODULE CODE HERE
\def\module{Module Title} %PUT YOPUR MODULE TITLE HERE

\lhead[Module Code: \code01]{Module Code: \code01} %Change 01 to 09 if this is a resit of a module that has changed syllabus
\rfoot[\bf Turn the page over]{\bf Turn the page over}
\lfoot[Page {\bf \thepage} of {\bf \pageref{LastPage}}]{Page {\bf \thepage} of
	{\bf \pageref{LastPage}}}


\newenvironment{sectionrubric}{\begin{itemize}\itemsep 0em \bf}{\end{itemize}}
\renewcommand{\headrulewidth}{0pt}
%%%%%%%%%%%%%%%%%%%%%%%%%%%%%%%%%%%%%%%%%%%%%%%%%%%%%%%%%%%%%%%%%%%
%%%%%%% Commands that are useful go here                    %%%%%%%%
%%%%%%%%%%%%%%%%%%%%%%%%%%%%%%%%%%%%%%%%%%%%%%%%%%%%%%%%%%%%%%%%%%%
\newcommand{\del}{\boldsymbol{\nabla}}
\newcommand{\curl}[1]{\del\times\vec{#1}}
\renewcommand{\div}[1]{\del\cdot\vec{#1}}
\newcommand{\uvec}[1]{\boldsymbol{\widehat{#1}}}
\newcommand{\vecg}[1]{{\boldsymbol{#1}}}
\newcommand{\vect}[1]{\ensuremath{\protect\underaccent{\tilde}{\boldsymbol{#1}}}}
\newcommand{\gtitle}[1]{{\sf #1}}
\newcommand{\note}[1]{[\textbf{#1}]}
\newcommand{\refnote}[1]{{\,{\textsl{\scriptsize [#1]}}}}
\newcommand{\kb}{k_{\scriptscriptstyle B}}
\newcommand{\kbT}{k_{\scriptscriptstyle B}T}
\newcommand{\concept}[2]{\noindent{\textbf{#1} {#2}}\\[5truept]}
\newcommand{\question}[2]{[{\bf Ex:} #1 \textsl{#2}]}
\newcommand{\ie}{\emph{i.e.\/}}
\newcommand{\etc}{\emph{etc\/}}
\newcommand{\eg}{\emph{e.g.\/}}
\newcommand{\Eg}{\emph{E.g.\/}}
\newcommand{\dimp}{d\!\!\!\hskip1truept^-}
%\newcommand{\pd}[3]{\left.\frac{\partial #1}{\partial #2}\right|_{#3}}
\newcommand{\pdd}[2]{\frac{\partial #1}{\partial #2}}
\newcommand{\units}[1]{\ensuremath{\,\mathrm{#1}}}
\newcommand{\hint}[1]{\textit{[#1]}}
\newcommand{\pts}[1]{\hspace*{0.1cm}\hfill[#1]}
\newcommand{\ptsmk}[1]{\phantom{.}\hfill[\textbf{#1~Marks}]}
\newcommand{\twocol}[4]{\parbox[t]{#1}{#3}\quad\parbox[t]{#2}{#4}}
\newcommand{\mean}[1]{\left\langle #1\right\rangle}
\newcommand{\Z}{{\cal Z\/}}
\renewcommand{\L}{{\cal L\/}}
\renewcommand{\H}{{\cal H\/}}
\newcommand{\ket}[1]{\left|#1\right.\rangle}
\newcommand{\bra}[1]{\langle\left.#1\right|}
\newcommand{\micron}{\mu\textrm{m}}
\renewcommand{\i}{\uvec{i}}
\renewcommand{\j}{\uvec{j}}
\renewcommand{\k}{\uvec{k}}
\renewcommand{\bar}[1]{\ensuremath{\overline{#1}}}
\newcommand{\conv}{\ensuremath{\,\circledast\,}}



\begin{document}
	
	\normalsize
	\vspace{3mm}
	
	{\renewcommand{\arraystretch}{1.5}
		\begin{tabular}[h]{p{9cm}p{7cm}}
			{\bf Module Title: \module}\vspace{2cm} %ENTER MODULE TITLE HERE
			& {\bf \copyright ~ UNIVERSITY OF LEEDS} \\
			\noindent{\bf School of Physics and Astronomy} & {\bf \semester ~ \session}\\ 
		\end{tabular}
	}
	
	\vspace*{1.5cm}
	
	
	\textbf{This is an open book assessment.} You may consult any of your own notes.
	\textbf{You must provide an explanation for all your answers in your own words.}
	This will vary from a few words to two to three sentences depending on the material. This will be to demonstrate your understanding of the course.
	
	\textbf{Do not just repeat answers from your notes without this explanation.}
	Make sure your method of calculation is clearly shown.  
	
	\textbf{If you make use of websites or textbooks to answer specific questions, you must list them at the end of the relevant answer.}
	
	\textbf{Assessment information:}
	
	\begin{itemize}
		\item This assessment is made up of \pageref{LastPage} pages.  
		\item \textbf{You must upload your answers viaGradeScope} to Minerva \textbf{within 48 hours of theassessmentbeing released.} You are advised to allow up to four hours to photograph your answers, and upload  as a PDF  toGradeScope. The upload link will be found in the Assessment section for each module on Minerva and will be available throughout the period of the assessment
		\item Although the upload is open for the full period of the assessment, you
		are advised that the assessment should only require 2 hours to complete.
		\item \textbf{Late submission of answers is not possible. }
		\item You must answer \textbf{all} of the questions in this assessment.
		\item You should cross out any work you do not want to be marked. 
		\item You should indicate the final answer to each question by underlining it. 
		\item As part of the process of submitted through GradeScope you must identify
		which questions are answered on which uploaded pages. You must also check
		that you have uploaded all the work you wish to be marked as part of this
		assessment and that the answers uploaded are clearly legible. Failure to do
		so may result in your work not being marked. 
		\item This is a formal University assessment. \textbf{You must not share or
			discuss any aspect of this assessment, your answers or the module more
			generally} with anyone whether a student or not during the period the
		assessment is open.
	\end{itemize}
	
	
	% IF USING A UNIX MACHINE YOU MIGHT NEED TO CHANGE {LastPage} to {lastpage}
	% IF USING A SECTION A & B FORMAT, REPEAT THE INSTRUCTIONS FOR THAT SECTION AT THE START OF THE SECTION.
	\newpage
	\noindent\textbf{Approximate values of some constants}\\[10truept]
	{\setstretch{1.25}
	\begin{tabular}{l@{\hspace{1.0truecm}}l}
		Speed of light in a vacuum, $c$ & $2.998\times10^8\units{m\,s}^{-1}$\\
		Electron Charge, $e$  & $1.602\times10^{-19} \units{C}$\\
		Electron rest mass,   $m_e$ & $9.11 \times10^{-31} \units{kg}=
		0.511\units{MeV\,c}^{-2}$\\
		Proton rest mass, $m_p$& $1.673 \times 10^{-27}\units{kg}   =  938.3 \units{MeV c}^{-2}$\\
		Unified atomic mass unit, $u$&   $1.661 \times 10^{-27} \units{kg}  =  931.494
		\units{MeV c}^{-2}$\\
		Fine structure constant, $\alpha$ & $1/137.036  $\\
		Planck constant, $h$ & $6.626 \times 10^{-34}\units{J\,s}$\\
		Boltzmann constant, $k_B$ & $1.381 \times 10^{-23}\units{J\,K}^{-1} =  8.617
		\times 10^{-5}\units{eV\,K}^{-1}$\\
		Coulomb constant, $\mathrm{k}=1/4\pi\epsilon_0$ & $8.987 \times 10^9\units{N\,m}^2\units{C}^{-2}$\\
		Rydberg constant, $R$& $1.09373 \times 10^7 \units{m}^{-1}$\\
		Avogadro constant, $N_A$& $6.022 \times 10^{23} \units{mol}^{-1}$\\
		Gas constant, $R$&$8.314\units{J\,K}^{-1}\units{mol}^{-1}$\\
		Stefan Boltzmann constant, $\sigma$ & $5.670 \times 10^{-8}\units{W\,m}^{-2}\units{K}^{-4}$\\
		Bohr magneton, $\mu_B$& $9.274 \times 10^{-24}\units{J\,T}^{-1}$\\
		Gravitational constant, $ G$& $6.673 \times
		10^{-11}\units{m}^3\units{kg}^{-1}\units{s}^{-2}$\\
		Acceleration due to gravity, $g$& $9.806\units{m\,s}^{-2}$\\
		Permeability of free space, $\mu_0$ & $4\pi\times  10^{-7}\units{H\,m}^{-1}$\\
		Permittivity of free space, $\epsilon_0$& $8.854 \times
		10^{-12}\units{F\,m}^{-1}$\\
		1 Parsec, pc   &$3.086 \times 10^{16}\units{m}$\\
		Solar mass, $M_{\odot}$& $1.99  \times 10^{30}\units{kg}$\\
		Magnetic flux quantum, $\Phi_0$&$2.0679\times 10^{-15}\units{Wb}$
	\end{tabular}
	\vspace{1.5cm}
	
	\noindent\textbf{Some SI prefixes}\\[10truept]
	\begin{tabular*}{15 cm}{l@{\extracolsep{1cm}}lllll}
		\textit{Multiple} & \textit{Prefix} & \textit{Symbol} &
		\textit{Multiple} & \textit{Prefix} & \textit{Symbol}\\
		$10^{-18}$&atto&a&
		$10^{-9}$&nano&n\\
		$10^{-15}$&femto&f&
		$10^9$&giga&G\\
		$10^{-12}$&pico&p&
		$10^{12}$&tera&T
	\end{tabular*}
}
	%%%%%%%%%%%%%%%%%%%%%%%%%%%%%%%%%%%%%%%%%%%%%%%%%%%%%%%%%%%%%%%%%%%
	\newpage
	
	
	{\large\textbf{SECTION A}}
	%%%%%%%%%%%%%%%%%%%%%%%%%%%%%%%%%%%%%%%%%%%%%%%%%%%%%%%%%%%%%%%%%%%
	%%%%%%%%%%  Adjust rubric to suit exam                      %%%%%%%
	%%%%%%%%%%%%%%%%%%%%%%%%%%%%%%%%%%%%%%%%%%%%%%%%%%%%%%%%%%%%%%%%%%%
	\begin{sectionrubric}
		\item You must answer all the questions from this section.
		\item This section is worth 20 marks.
		\item You are advised to spend 30 minutes on this section.
	\end{sectionrubric}
	%%%%%%%%%%%%%%%%%%%%%%%%%%%%%%%%%%%%%%%%%%%%%%%%%%%%%%%%%%%%%%%%%%%
	\renewcommand{\theenumi}{A\arabic{enumi}}
	%%%%%%%%%   Section A questions here                       %%%%%%%%
	\begin{enumerate}
		%%%%%%%%%%%%%%%%%%%%%%%%%%%%%%%%%%%%%%%%%%%%%%%%%%%%%%%%%%%%%%%%%%%
		\item
		Q1.  \ptsmk{7}
		
		\item
		Q2 etc
		\ptsmk{7}
		
		
		
		
		
		
		
		%%%%%%%%%%%%%%%%%%%%%%%%%%%%%%%%%%%%%%%%%%%%%%%%%%%%%%%%%%%%%%%%%%%
	\end{enumerate}
	%%%%%%%%%%%%%%%%%%%%%%%%%%%%%%%%%%%%%%%%%%%%%%%%%%%%%%%%%%%%%%%%%%%
	\newpage    %%%%%%%%      Section B Starts Here %%%%%%%%%%%%%%%%%%%
	%%%%%%%%%%%%%%%%%%%%%%%%%%%%%%%%%%%%%%%%%%%%%%%%%%%%%%%%%%%%%%%%%%%
	\vskip1.0truecm
	%
	{\large\textbf{SECTION B}}
	%%%%%%%%%%%%%%%%%%%%%%%%%%%%%%%%%%%%%%%%%%%%%%%%%%%%%%%%%%%%%%%%%%%
	%%%%%%%%%%  Adjust rubric to suit exam                      %%%%%%%
	%%%%%%%%%%%%%%%%%%%%%%%%%%%%%%%%%%%%%%%%%%%%%%%%%%%%%%%%%%%%%%%%%%%
	\begin{sectionrubric}
	\item You must answer all the questions from this section.
	\item This section is worth 60 marks.
	\item You are advised to spend 90 minutes on this section.
	\end{sectionrubric}
	%%%%%%%%%%%%%%%%%%%%%%%%%%%%%%%%%%%%%%%%%%%%%%%%%%%%%%%%%%%%%%%%%%%
	\renewcommand{\theenumi}{B\arabic{enumi}}
	%%%%%%%%%%%%%%%%%%%%%%%%%%%%%%%%%%%%%%%%%%%%%%%%%%%%%%%%%%%%%%%%%%%
	\begin{enumerate}
		%%%%%%%%%%%%%%%%%%%%%%%%%%%%%%%%%%%%%%%%%%%%%%%%%%%%%%%%%%%%%%%%%%
		
		
		%%%%%%%%%%%%%%%%%%%%%%%%%%%%%%%%%%%%%%%%%%%%%%%%%%%%%%%%%%%%%%%%%%
		%\bigskip%%%%%%%%% 
		%%%%%%%%%%%%%%%%%%%%%%%%%%%%%%%%%%%%%%%%%%%%%%%%%%%%%%%%%%%%%%%%%%
		%%%  CMB
		%%%%%%%%%%%%%%%%%%%%%%%%%%%%%%%%%%%%%%%%%%%%%%%%%%%%%%%%%%%%%%%%%%%%%%%%%%%%%%%%
		\item 
		\begin{enumerate}
			
			\item
			
			Part a here
			\pts{4}
			
			
			\item
			
			Part b
			\pts{8}
			
			\item
			Part c
			\pts{5}
			
			
			
			
			\item 
			part d
			\pts{13}
			
		\end{enumerate}
		\ptsmk{20}
		
		% use spacing to create gaps between questions (or ideally one per page).
		% either:
		%\vspace*{1in}
		% or
		\newpage
		
		\item
		\begin{enumerate}
			
			\item
			
			Next Q part a etc
			\pts{20}
		\end{enumerate}
		\ptsmk{20}
		% use spacing to create gaps between questions (or ideally one per page).
		% either:
		%\vspace*{1in}
		% or
		\newpage
		
		\item
		\begin{enumerate}
			
			\item
			
			Next Q part a etc
			\pts{20}
		\end{enumerate}
		\ptsmk{20}
		
		
	\end{enumerate}
	
	
	\rfoot[\bf End.]{\bf End.}
	
	%IF A FORMULA SHEET IS ATTACHED, PLEASE USE END OF QUESTIONS HERE AND THEN END OF PAPER AFTER THE FORMULA SHEET.
	
\end{document}




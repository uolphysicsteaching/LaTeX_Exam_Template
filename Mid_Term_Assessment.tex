\documentclass[12pt]{article}
% GIT Repository for this file is at:
% https://github.com/uolphysicsteaching/LaTeX_Exam_Template
\usepackage{fancyhdr,a4wide}
\usepackage[mathscr]{eucal}
\usepackage[T1]{fontenc}
\usepackage{uarial}
\renewcommand{\familydefault}{\sfdefault}
\usepackage[T1]{fontenc}
\usepackage{blindtext}
\usepackage{graphicx}
\usepackage{amscd}
%\usepackage{xypic}
\usepackage{amsfonts}
\usepackage{amsmath}
\usepackage{amssymb}
%\usepackage{latexsymb}
\usepackage{enumitem}
\usepackage{lastpage}
\setlength{\headheight}{15.2pt}
\pagestyle{fancy}
\setlength{\textwidth}{16truecm}
\setlength{\textheight}{24truecm}
\setlength{\oddsidemargin}{3mm}
\setlength{\evensidemargin}{3mm}
\setlength{\topmargin}{1mm}
\usepackage{setspace}
%\usepackage{cmbright} %PUTS TEXT AND EQUATIONS INTO SANS SERIF FONT

\fancyhf{}
\hoffset=-4mm
\voffset=-20mm
\parindent=0mm
\parskip=2mm

%%%%%%%%%%%%%%%%%%%%%%%%%%%%%%%%%%%%%%%%%%%%%%%%%%%%%%%%%%%%%%%%%%%%%%%%%%%
%%%%%%%%%%%%% Module Leader Editable Settings %%%%%%%%%%%%%%%%%%%%%%%%%%%%%
%%%%%%%%%%%%%%%%%%%%%%%%%%%%%%%%%%%%%%%%%%%%%%%%%%%%%%%%%%%%%%%%%%%%%%%%%%

\def\semester{Semester 1}
\def\session{2021/2022}
\def\paperno{01} % 09 FOR A MODULE WITH AN OLD SYLLABUS
\def\code{PHYSXXXX} %PUT CORRECT MODULE CODE HERE
\def\module{Module Title} %PUT YOPUR MODULE TITLE HERE
\def\totaltime{1 hour and 30 minutes\ } %ANSWER TIME + UPLOAD TIME
\def\answertime{1 hour\ } % 1.5 MINUTES PER MARK
\def\uploadtime{30 minutes}% 30 MINUTES UNLESS LONGER NEEDED
\def\paperweight{20\%\ }


%%%%%%%%%%%%%%%%%%%%%%%%%%%%%%%%%%%%%%%%%%%%%%%%%%%%%%%%%%%%%%%%%%%
%%%%%%% Commands that are useful go here                    %%%%%%%%
%%%%%%%%%%%%%%%%%%%%%%%%%%%%%%%%%%%%%%%%%%%%%%%%%%%%%%%%%%%%%%%%%%%
\lhead[Module Code: \code01]{Module Code: \code\paperno} 
\rfoot[\bf Turn the page over]{\bf Turn the page over}
\lfoot[Page {\bf \thepage} of {\bf \pageref{LastPage}}]{Page {\bf \thepage} of
	{\bf \pageref{LastPage}}}

\newenvironment{sectionrubric}{\begin{itemize}\itemsep 0em \bf}{\end{itemize}}
\renewcommand{\headrulewidth}{0pt}
\newcommand{\del}{\boldsymbol{\nabla}}
\newcommand{\curl}[1]{\del\times\vec{#1}}
\renewcommand{\div}[1]{\del\cdot\vec{#1}}
\newcommand{\uvec}[1]{\boldsymbol{\widehat{#1}}}
\newcommand{\vecg}[1]{{\boldsymbol{#1}}}
\newcommand{\vect}[1]{\ensuremath{\protect\underaccent{\tilde}{\boldsymbol{#1}}}}
\newcommand{\gtitle}[1]{{\sf #1}}
\newcommand{\note}[1]{[\textbf{#1}]}
\newcommand{\refnote}[1]{{\,{\textsl{\scriptsize [#1]}}}}
\newcommand{\kb}{k_{\scriptscriptstyle B}}
\newcommand{\kbT}{k_{\scriptscriptstyle B}T}
\newcommand{\concept}[2]{\noindent{\textbf{#1} {#2}}\\[5truept]}
\newcommand{\question}[2]{[{\bf Ex:} #1 \textsl{#2}]}
\newcommand{\ie}{\emph{i.e.\/}}
\newcommand{\etc}{\emph{etc\/}}
\newcommand{\eg}{\emph{e.g.\/}}
\newcommand{\Eg}{\emph{E.g.\/}}
\newcommand{\dimp}{d\!\!\!\hskip1truept^-}
%\newcommand{\pd}[3]{\left.\frac{\partial #1}{\partial #2}\right|_{#3}}
\newcommand{\pdd}[2]{\frac{\partial #1}{\partial #2}}
\newcommand{\units}[1]{\ensuremath{\,\mathrm{#1}}}
\newcommand{\hint}[1]{\textit{[#1]}}
\newcommand{\pts}[1]{\hspace*{0.1cm}\hfill[#1]}
\newcommand{\ptsmk}[1]{\phantom{.}\hfill[\textbf{#1~Marks}]}
\newcommand{\twocol}[4]{\parbox[t]{#1}{#3}\quad\parbox[t]{#2}{#4}}
\newcommand{\mean}[1]{\left\langle #1\right\rangle}
\newcommand{\Z}{{\cal Z\/}}
\renewcommand{\L}{{\cal L\/}}
\renewcommand{\H}{{\cal H\/}}
\newcommand{\ket}[1]{\left|#1\right.\rangle}
\newcommand{\bra}[1]{\langle\left.#1\right|}
\newcommand{\micron}{\mu\textrm{m}}
\renewcommand{\i}{\uvec{i}}
\renewcommand{\j}{\uvec{j}}
\renewcommand{\k}{\uvec{k}}
\renewcommand{\bar}[1]{\ensuremath{\overline{#1}}}
\newcommand{\conv}{\ensuremath{\,\circledast\,}}
\newcommand{\chem}[1]{\ensuremath{\mathrm{#1}}}
%
% Imports an eps file to make a figure
%
% \exfig[Filename_prefix]{figure number} imports a file called Filename_prefix-#.eps
% Default is just Figs-#.eps e.g. \exfig{2} includes "Figs-2.eps"
%
\newcommand{\exfig}[2][Figs]{
	\hspace{0.1cm}\newline
	\begin{center}
		\includegraphics[max height=\textheight,max width=\textwidth]{#1-#2}
	\end{center}
}

\begin{document}
	
	\normalsize
	\vspace{3mm}
	
	{\renewcommand{\arraystretch}{1.5}
		\begin{tabular}[h]{p{9cm}p{7cm}}
			{\bf Module Title: \module}\vspace{1.5cm} %ENTER MODULE TITLE HERE
			& {\bf \copyright ~ UNIVERSITY OF LEEDS} \\
			& \textbf{Mid-Term Paper}\\
			\noindent{\bf School of Physics and Astronomy} & {\bf \semester ~ \session}\\ 
		\end{tabular}
	}
	
	\vspace*{1cm}
	
	
\textbf{Calculator instructions:}
\begin{itemize}[itemsep=3pt,topsep=0pt]
\item You are allowed to use a calculator or a computer calculator in this assessment. 
\end{itemize}
\textbf{Dictionary instructions:}
\begin{itemize}[itemsep=3pt,topsep=0pt]
	\item You are allowed to use your own dictionary in this assessment and/or the Spell Checker facility on your computer. 
\end{itemize}
\textbf{Assessment Information:}
\begin{itemize}[itemsep=0pt,topsep=0pt]
\item There are \pageref{LastPage} pages to this online assessment.
\item You will have \textbf{\totaltime} to complete the assessment and upload your answers to Gradescope.
\item You are recommended to take a \textbf{maximum of \answertime} within the time available to answer the questions and the remaining \uploadtime to upload your answers. 
\item This assessment is worth \paperweight of the overall module mark
\item Answer \textbf{all} of the questions in \textbf{all} of the sections of this paper.
\item You must submit your answers via Minerva to Gradescope. You will find the link for uploading your work in the Assessment section of the module pages on Minerva in the same folder as you downloaded this paper from.
\item Please include your Student Identification Number (SID) at the top of each page of your answers. You do not need to include your name.
\item When submitting your work, you must identify which questions are answered on which uploaded pages. You must also check that you have uploaded all the work you wish to be marked as part of this assessment and that the answers uploaded are clearly legible. Failure to do so may result in your work not being marked.
\item If there is anything that needs clarification about the questions in this paper, please contact the module leader by email and cc the Physics Exams team \textbf{physicsexams@leeds.ac.uk} and we will respond to you as quickly as possible within normal working hours UK time (9:00-17:00 hours, Monday-Friday). 
\item If you have any technical difficulties please contact the Physics Exams team at the address above \textbf{before the deadline for submission}.
\item \textbf{This is a formal University assessment.} You must not share or discuss any aspect of this assessment, your answers or the module more generally with anyone whether a student or not during the period the assessment is open, with the exception of the module leader and Physics exams team.
\end{itemize}	
	
	
	% IF USING A UNIX MACHINE YOU MIGHT NEED TO CHANGE {LastPage} to {lastpage}
	% IF USING A SECTION A & B FORMAT, REPEAT THE INSTRUCTIONS FOR THAT SECTION AT THE START OF THE SECTION.
	\newpage
	\noindent\textbf{Approximate values of some constants}\\[10truept]
	{\setstretch{1.25}
	\begin{tabular}{l@{\hspace{1.0truecm}}l}
		Speed of light in a vacuum, $c$ & $2.998\times10^8\units{m\,s}^{-1}$\\
		Electron Charge, $e$  & $1.602\times10^{-19} \units{C}$\\
		Electron rest mass,   $m_e$ & $9.11 \times10^{-31} \units{kg}=
		0.511\units{MeV\,c}^{-2}$\\
		Proton rest mass, $m_p$& $1.673 \times 10^{-27}\units{kg}   =  938.3 \units{MeV c}^{-2}$\\
		Unified atomic mass unit, $u$&   $1.661 \times 10^{-27} \units{kg}  =  931.494
		\units{MeV c}^{-2}$\\
		Fine structure constant, $\alpha$ & $1/137.036  $\\
		Planck constant, $h$ & $6.626 \times 10^{-34}\units{J\,s}$\\
		Boltzmann constant, $k_B$ & $1.381 \times 10^{-23}\units{J\,K}^{-1} =  8.617
		\times 10^{-5}\units{eV\,K}^{-1}$\\
		Coulomb constant, $\mathrm{k}=1/4\pi\epsilon_0$ & $8.987 \times 10^9\units{N\,m}^2\units{C}^{-2}$\\
		Rydberg constant, $R$& $1.09373 \times 10^7 \units{m}^{-1}$\\
		Avogadro constant, $N_A$& $6.022 \times 10^{23} \units{mol}^{-1}$\\
		Gas constant, $R$&$8.314\units{J\,K}^{-1}\units{mol}^{-1}$\\
		Stefan Boltzmann constant, $\sigma$ & $5.670 \times 10^{-8}\units{W\,m}^{-2}\units{K}^{-4}$\\
		Bohr magneton, $\mu_B$& $9.274 \times 10^{-24}\units{J\,T}^{-1}$\\
		Gravitational constant, $ G$& $6.673 \times
		10^{-11}\units{m}^3\units{kg}^{-1}\units{s}^{-2}$\\
		Acceleration due to gravity, $g$& $9.806\units{m\,s}^{-2}$\\
		Permeability of free space, $\mu_0$ & $4\pi\times  10^{-7}\units{H\,m}^{-1}$\\
		Permittivity of free space, $\epsilon_0$& $8.854 \times
		10^{-12}\units{F\,m}^{-1}$\\
		1 Parsec, pc   &$3.086 \times 10^{16}\units{m}$\\
		Solar mass, $M_{\odot}$& $1.99  \times 10^{30}\units{kg}$\\
		Magnetic flux quantum, $\Phi_0$&$2.0679\times 10^{-15}\units{Wb}$
	\end{tabular}
	\vspace{1.5cm}
	
	\noindent\textbf{Some SI prefixes}\\[10truept]
	\begin{tabular*}{15 cm}{l@{\extracolsep{1cm}}lllll}
		\textit{Multiple} & \textit{Prefix} & \textit{Symbol} &
		\textit{Multiple} & \textit{Prefix} & \textit{Symbol}\\
		$10^{-18}$&atto&a&
		$10^{-9}$&nano&n\\
		$10^{-15}$&femto&f&
		$10^9$&giga&G\\
		$10^{-12}$&pico&p&
		$10^{12}$&tera&T
	\end{tabular*}
}
	%%%%%%%%%%%%%%%%%%%%%%%%%%%%%%%%%%%%%%%%%%%%%%%%%%%%%%%%%%%%%%%%%%%
	\newpage
	
	
	%%%%%%%%%%%%%%%%%%%%%%%%%%%%%%%%%%%%%%%%%%%%%%%%%%%%%%%%%%%%%%%%%%%
	%%%%%%%%%%  Adjust rubric to suit exam                      %%%%%%%
	%%%%%%%%%%%%%%%%%%%%%%%%%%%%%%%%%%%%%%%%%%%%%%%%%%%%%%%%%%%%%%%%%%%
	\begin{sectionrubric}
		\item This paper is worth 40 marks.
		\item You must answer all the questions.
		\item You are advised to spend \answertime on this paper.
	\end{sectionrubric}
	%%%%%%%%%%%%%%%%%%%%%%%%%%%%%%%%%%%%%%%%%%%%%%%%%%%%%%%%%%%%%%%%%%%
	\renewcommand{\theenumi}{\arabic{enumi}}
	\begin{enumerate}
	%%%%%%%%%%%%%%%%%%%%%%%%%%%%%%%%%%%%%%%%%%%%%%%%%%%%%%%%%%%%%%%%%%%
%You can just write every question as a separate \item and the numbering will be sorted for you
%You just indicate the number of marks for each question with \ptsmk{xxx}
		
\item 1.	Mid-term papers only have ``Section A'' style questions which should test the breadth of the student’s knowledge from weeks 1 to 5. Each question should be ``standalone'' in that they do not depend on a previous question in order for students to understand what they need to do. Questions should not include unseen problems 
\ptsmk{2}

\item Questions should be worth between 3 and 8 marks and not have multiple parts to them. Wherever possible ensure that students are required to provide eplanation or interpretation of their answer and/or produce some diagram as this makes it significantly easier to demonstrate collusion.
\ptsmk{5}

		\item Try to write the questions as simply as possible and avoid convoluted sentence structures. Avoid setting scenarios that include the student answering the problem – i.e. write “A person is in a balloon when….” and not “You are in a balloon when...
\ptsmk{6}


\item Only use \textbf{bold} for emphasis, do not use \underline{underline} or \textit{italics} or any other textual marks (including in rubric and instructions!)
\ptsmk{4}

\item Text should all be Arial font, 12pt. Try to avoid the use of non-standard symbol fonts where possible as they can cause problems if they do not embed in the pdf correctly. Equations do not need to be sans-serif especially if it makes them less clear e.g. confusion between the digit $1$ and the lower case letter $l$. %By default the TeX template is using serif font in equations.
\ptsmk{7}


	%%%%%%%%%%%%%%%%%%%%%%%%%%%%%%%%%%%%%%%%%%%%%%%%%%%%%%%%%%%%%%%%%%%
	\end{enumerate}

	
	
	\rfoot[\bf End.]{\bf End.}
	
	%IF A FORMULA SHEET IS ATTACHED, PLEASE USE END OF QUESTIONS HERE AND THEN END OF PAPER AFTER THE FORMULA SHEET.
	
\end{document}



